\documentclass[mathserif,handout]{beamer}
%\documentclass{beamer}
\usetheme{Warsaw}
\usecolortheme{seahorse}
\usecolortheme{orchid}
\usepackage{amsmath,verbatim}
\usepackage{listings}
\usepackage[english]{babel}
\setbeamercovered{transparent}

\newcommand{\Deltap}{\ensuremath{\Delta^{\!+}}}
\newcommand{\trans}{\ensuremath{{}^\mathrm{T}}}
\newcommand{\eps}{\varepsilon}
\newcommand*{\approxdist}{\mathrel{\vcenter{\offinterlineskip
\vskip-.25ex\hbox{\hskip.55ex$\cdot$}\vskip-.25ex\hbox{$\sim$}
\vskip-.5ex\hbox{\hskip.55ex$\cdot$}}}}

\lstdefinelanguage{myR}
{
   language=R,
   otherkeywords={read.table, set.seed, head},
   deletekeywords={url,codes, t, dt, Call, formula,Q, R, on,by,hat,is,
col, set,start,end,deltat,zip},
   sensitive=true,
   breaklines=true,
   morecomment=[l]{\#},
   morestring=[b]",
   morestring=[b]',
   basicstyle =\ttfamily\small,
   keywordstyle=\bfseries,
   showtabs=false,
   showstringspaces=false,
   literate= {~}{$\sim$}{2},
   numberstyle=\sffamily\scriptsize,
   stepnumber=2
 }

\lstset{basicstyle=\ttfamily\color{blue}}

\begin{document}

\title{On-line algorithms for IoT data streams}
\author[Darren Wilkinson --- IoT Systems, 16/1/17]{\textbf{\large Darren Wilkinson} \\
\url{@darrenjw}\\
\alert{\url{http://tinyurl.com/darrenjw}}\\
School of Mathematics \& Statistics\\Newcastle University, UK}
\date{IoT Systems Workshop\\Lumley Castle, Co. Durham\\ 16th--17th January 2017}

\frame{\titlepage}

\frame{
  \frametitle{Statistical analysis of streaming IoT data}
  \begin{itemize}
  \item IoT is really all about the \alert{data} --- data is not ``smart'', but the analysis of it could be...
  \item Some statistical analysis must be done \alert{off-line}, using algorithms that must make many passes over the data, but many algorithms can be implemented in an \alert{on-line} fashion, \alert{sequentially} processing each unit of data as it arrives via a data \alert{stream}
  \item \alert{State space models} and \alert{partially observed Markov process (POMP) models} provide a foundation for understanding (noisy) time series data from sensors
    \item \alert{Bayesian inference} approaches to state updating in state space models can often be carried out on-line...
  \end{itemize}
}

\frame{
  \frametitle{Abstraction for streaming data processing}
  \begin{itemize}
  \item Fundamental streaming data abstraction:\\
    \alert{\texttt{update: (State, Observation) => State}}
  \item Pure function to combine current knowledge of world state with a new data observation to get an updated view of current world state
  \item Can be applied to any kind of \alert{\texttt{Stream[Observation]}} data structure via a HOF such as \alert{\texttt{foldLeft}} or \alert{\texttt{scanLeft}} to obtain a \alert{\texttt{Stream[State]}}
  \item Consequently, a function corresponding to this type signature can be dropped into essentially any kind of streaming data framework: Storm, Kafka, Akka-streams, Flink, etc., without needing to couple the updating logic to any particular stream technology
  \item Given a decision function \alert{\texttt{action: State => Decision}}, can \alert{map} over a \alert{\texttt{Stream[State]}} to get a \alert{\texttt{Stream[Decision]}}
  %  \item The ability to capture updating logic in a function of this type is precisely what we mean by saying that an algorithm can be implemented on-line
    \end{itemize}
  }

\frame{
  \frametitle{PhD Students}
  \begin{itemize}
  \item \alert{Rui Viera} (sponsored by RedHat)
    \begin{itemize}
    \item Review, comparison and development of on-line algorithms for Bayesian inference for state space models of streaming internet time series data
      \item Implementation in R (with C++), Java and Scala
      \end{itemize}
  \item \alert{Jonny Law} (EPSRC CDT in CCfBD)
    \begin{itemize}
    \item Extension of Rui's work to irregularly spaced (time-stamped) data streams
    \item Focus on functional programming approaches using Scala, building complex models by composing from simple components, and coupling to the Akka-streams reactive-streams library
      \item Application to Newcastle's Urban Observatory data
      \end{itemize}
    \end{itemize}
  }


\begin{frame}
  \frametitle{Links and further info:}
  \begin{itemize}
  \item \alert{Rui}:
    \begin{itemize}
    \item \texttt{@ruimvieira} on Twitter and \texttt{@ruivieira} on GitHub
    \item Blog: \url{www.ruivieira.org}
      \item Paper: \alert{\url{arxiv.org/abs/1608.08666}} (review and comparison of on-line inference algorithms)
    \end{itemize}
  \item \alert{Jonny}:
    \begin{itemize}
    \item \texttt{@lawsy} on Twitter and \texttt{@jonnylaw} on GitHub
      \item Blog: \url{jonnylaw.github.io}
      \item Paper and Software: \alert{\url{git.io/statespace}} (composable state-space models, embedding in akka-streams)
    \end{itemize}
  \item \alert{Me}:
    \begin{itemize}
    \item \texttt{@darrenjw} on Twitter and GitHub
    \item Blogs:
      \begin{itemize}
      \item \url{darrenjw.wordpress.com} --- mainly Scala for data science
      \item \url{darrenjw2.wordpress.com} --- mainly Raspberry Pi stuff
      \end{itemize}
    \end{itemize}
  \end{itemize}  
\end{frame}









\end{document}

